% vim:encoding=utf8 ft=tex sts=2 sw=2 et:

\documentclass{classrep}
\usepackage[utf8]{inputenc}
\usepackage{color}
\usepackage{mathtools}

\usepackage{subfig}
\usepackage{float}

\usepackage[labelfont=it]{caption}

\studycycle{Informatyka, studia niestacjonarne, mgr II st.}
\coursesemester{I}

\coursename{Przetwarzanie obrazu i dźwięku}
\courseyear{2015/2016}

\courseteacher{mgr inż. Piotr Ożdżyński}
\coursegroup{Sobota, 14:15}

\author{
  \studentinfo{Jakub Antosik}{206788} \and
  \studentinfo{Andrzej Lisowski}{206807} 
}

\title{Zadanie 3: Analiza częstotliwości podstawowej dźwięku.}

\begin{document}
\maketitle

\section{Cel}
Celem zadania było zapoznanie się z metodami analizy dźwięku, a w szczególności znajdowania okresu i częstotliwości. Badane były dwie grupy metod: realizowane w dziedzinie czasu oraz w dziedzinie częstotliwości. W części implementacyjnej należało stworzyć program w wybranym przez siebie języku programowania, który będzie w stanie przeprowadzić po jednej, wybranej operacji operacji z z każdej z grup. Do tego celu wykorzystano szkielet apliakcji z zadań 1 i 2.\\
\indent
Szczegółowy opis zadania został przedstawiony w [1].

\section{Wprowadzenie}
//TODO

\subsection{Autokorelacja}
//TODO

\subsection{Analiza widma Fouriera sygnału}
//TODO

\section{Opis implementacji}
Opis implementacji został przedstawiony w sprawozdaniu do zadania 1 [2]. Zadanie 3 zostało zrealizowane poprzez rozszerzenie funkcjonalności programu. Dodany został nowy interfejs graficzny, dedykowany dla przetwarzania dźwięku oraz analizowane metody tj. autokorelację oraz analizę widma Fouriera sygnału.

\section{Materiały i metody}
Do aplikacji dodano szereg testowych dźwięków w celu dokładnej analizy badanych metod. Ich spis zamieszczono poniżej:\\
\indent
\begin{itemize}
\item Sztuczne
\begin{itemize}
\item Łatwe: 100Hz, 150Hz, 225Hz, 337Hz, 506Hz, 759Hz, 1139Hz, 1708Hz
\item Średnie: 90Hz, 135Hz, 202Hz, 303Hz, 455Hz, 683Hz, 1025Hz, 1537Hz
\item Trudne: 80Hz, 120Hz, 180Hz, 270Hz, 405Hz, 607Hz, 911Hz, 1366Hz
\end{itemize}
\item Naturalne
\begin{itemize}
\item Flet: 276Hz, 443Hz, 591Hz, 887Hz, 1265Hz, 1779Hz
\item Altówka: 130Hz, 196Hz, 247Hz, 294Hz, 369Hz, 440Hz, 698Hz
\end{itemize}
\item Sekwencje
\begin{itemize}
\item DWK altówka
\item KDF pianino
\end{itemize}
\end{itemize}

\section{Wyniki}
Sekcja prezentuje wyniki przeprowadzanego badania metody autokorelacji i analizy widma Fouriera sygnału. 

\subsection{Autokorelacja}
W poniższej tabeli przedstawiono częstotliwości badanych dźwięków - faktyczną oraz znalezioną w wyniku autokorelacji.\\
\\
\begin{tabular}{ l | c | c }
  \hline
  Dźwięk testowy & Autokorelacja & Analiza widma Fouriera \\
  \hline			
  Sztuczne, łatwe, 100Hz & TODO & TODO \\
  Sztuczne, łatwe, 150Hz & TODO & TODO \\
  Sztuczne, łatwe, 225Hz & TODO & TODO \\
  Sztuczne, łatwe, 337Hz & TODO & TODO \\
  Sztuczne, łatwe, 506Hz & TODO & TODO \\
  Sztuczne, łatwe, 759Hz & TODO & TODO \\
  Sztuczne, łatwe, 1139Hz & TODO & TODO \\
  Sztuczne, łatwe, 1708Hz & TODO & TODO \\
  \hline
  Sztuczne, średnie, 90Hz & TODO & TODO \\
  Sztuczne, średnie, 135Hz & TODO & TODO \\
  Sztuczne, średnie, 202Hz & TODO & TODO \\
  Sztuczne, średnie, 303Hz & TODO & TODO \\
  Sztuczne, średnie, 455Hz & TODO & TODO \\
  Sztuczne, średnie, 683Hz & TODO & TODO \\
  Sztuczne, średnie, 1025Hz & TODO & TODO \\
  Sztuczne, średnie, 1537Hz & TODO & TODO \\
  \hline 
  Sztuczne, trudne, 80Hz & TODO & TODO \\
  Sztuczne, trudne, 120Hz & TODO & TODO \\
  Sztuczne, trudne, 180Hz & TODO & TODO \\
  Sztuczne, trudne, 270Hz & TODO & TODO \\
  Sztuczne, trudne, 405Hz & TODO & TODO \\
  Sztuczne, trudne, 607Hz & TODO & TODO \\
  Sztuczne, trudne, 911Hz & TODO & TODO \\
  Sztuczne, trudne, 1366Hz & TODO & TODO \\
  \hline 
  Naturalne, flet, 276Hz & TODO & TODO \\
  Naturalne, flet, 443Hz & TODO & TODO \\
  Naturalne, flet, 591Hz & TODO & TODO \\
  Naturalne, flet, 887Hz & TODO & TODO \\
  Naturalne, flet, 1265Hz & TODO & TODO \\
  Naturalne, flet, 1779Hz & TODO & TODO \\
  \hline 
  Naturalne, altówka, 130Hz & TODO & TODO \\
  Naturalne, altówka, 196Hz & TODO & TODO \\
  Naturalne, altówka, 247Hz & TODO & TODO \\
  Naturalne, altówka, 294Hz & TODO & TODO \\
  Naturalne, altówka, 369Hz & TODO & TODO \\
  Naturalne, altówka, 440Hz & TODO & TODO \\
  Naturalne, altówka, 698Hz & TODO & TODO \\
  \hline 
\end{tabular}

\subsection{Analiza widma Fouriera sygnału}
Sekcja ta prezentuje wyniki wyniki zastosowania segmentacji dla obrazu 24-bitowego Girl. Zaprezentowane są nałożone maski dla różnych parametrów progu i minimalnej ilości pikseli na obszar.

\section{Dyskusja i wnioski}
//TODO

\begin{thebibliography}{1}
\bibitem{instruction_pol}\text{$http://ftims.edu.p.lodz.pl/pluginfile.php/20101/mod\_resource/content/1/$}\\
\text{$Third2012.pdf, 2015$}
\bibitem{instruction_pol}\text{$https://github.com/alisowsk/image-and-sound-processing/blob/master/$}\\
\text{$sprawozdanie/sprawozdanie.pdf, 2015$}
\end{thebibliography}

\end{document}
