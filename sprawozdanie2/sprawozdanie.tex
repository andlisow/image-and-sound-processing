% vim:encoding=utf8 ft=tex sts=2 sw=2 et:

\documentclass{classrep}
\usepackage[utf8]{inputenc}
\usepackage{color}
\usepackage{mathtools}

\usepackage{subfig}
\usepackage{float}

\usepackage[labelfont=it]{caption}

\studycycle{Informatyka, studia niestacjonarne, mgr II st.}
\coursesemester{I}

\coursename{Przetwarzanie obrazu i dźwięku}
\courseyear{2015/2016}

\courseteacher{mgr inż. Piotr Ożdżyński}
\coursegroup{Sobota, 14:15}

\author{
  \studentinfo{Jakub Antosik}{206788} \and
  \studentinfo{Andrzej Lisowski}{206807} 
}

\title{Zadanie 2: Filtracja w dziedzinie częstotliwości i segmentacja obrazu.}

\begin{document}
\maketitle

\section{Cel}
Celem zadania było zapoznanie się z transformatą Fouriera, filtracją w dziedzinie częstotliwości oraz segmentacją obrazu. W części implementacyjnej należało stworzyć program w wybranym przez siebie języku programowania, który będzie w stanie przeprowadzić analizowane operacje. W tym celu, wykorzystano aplikację z zadania 1.\\
Szczegółowy opis zadania został przedstawiony w [1].

\section{Wprowadzenie}
//TODO

\subsection{Transformata Fouriera i odwrotna transformata Fouriera}
//TODO

\subsection{Szybka transformata Fouriera i odwrotna szybka transformata Fouriera}
//TODO

\subsection{Filtracja}
//TODO

\subsection{Segmentacja}
//TODO

\section{Opis implementacji}
Opis implementacji został przedstawiony w sprawozdaniu do zadania 1 [3]. Zadanie 2 zostało zrealizowane poprzez rozszerzenie funkcjonalności programu o dodatkowe metody - transformatę Fouriera, filtracje w dziedzinie częstotliwości oraz segmentację obrazów.

\section{Materiały i metody}
Opis materiałów został przedstawiony w sprawozdaniu do zadania 1 [3]. Dodatkowe obrazy, użyte w celu analizy filtru z detekcją krawędzi są przedstawione poniżej.
//TODO

\section{Wyniki}
//TODO

\subsection{Szybka transformata Fouriera i odwrotna szybka transformata Fouriera}
Poniżej przedstawione zostały widma mocy i widma fazy dla wybranych obrazów 8- i 24-bitowych.

\begin{figure}[H]%
    \centering
    \subfloat[Widmo mocy]{{\includegraphics[width=3cm,height=3cm,keepaspectratio]{img/transformed/fft/abs_lena_8.png}}}%
    \qquad
    \qquad
    \subfloat[Widmoo fazy]{{\includegraphics[width=3cm,height=3cm,keepaspectratio]{img/transformed/fft/phase_lena_8.png}}}%
    \qquad
    \qquad
    \subfloat[Lena 8b]{{\includegraphics[width=3cm,height=3cm,keepaspectratio]{img/transformed/fft/lena_8.png}}}%
    \qquad
    \qquad
    \caption{Widmo mocy i widmo fazy dla obrazu Lena 8-bitowego.}%
\end{figure} 

\begin{figure}[H]%
    \centering
    \subfloat[Widmo mocy R]{{\includegraphics[width=3cm,height=3cm,keepaspectratio]{img/transformed/fft/abs_red_girl_24.png}}}%
    \qquad
    \qquad
    \subfloat[Widmo mocy G]{{\includegraphics[width=3cm,height=3cm,keepaspectratio]{img/transformed/fft/abs_green_girl_24.png}}}%    
    \qquad
    \qquad
    \subfloat[Widmo mocy B]{{\includegraphics[width=3cm,height=3cm,keepaspectratio]{img/transformed/fft/abs_blue_girl_24.png}}}%    
    \qquad   
    \qquad 
    \subfloat[Widmo fazy R]{{\includegraphics[width=3cm,height=3cm,keepaspectratio]{img/transformed/fft/phase_red_girl_24.png}}}%
    \qquad
    \qquad
    \subfloat[Widmo fazy G]{{\includegraphics[width=3cm,height=3cm,keepaspectratio]{img/transformed/fft/phase_green_girl_24.png}}}%
    \qquad
    \qquad
    \subfloat[Widmo fazy B]{{\includegraphics[width=3cm,height=3cm,keepaspectratio]{img/transformed/fft/phase_blue_girl_24.png}}}%
    \qquad
    \qquad
    \subfloat[Girl 24b]{{\includegraphics[width=3cm,height=3cm,keepaspectratio]{img/transformed/fft/girl_24.png}}}%
    \qquad
    \caption{Widma mocy i widma fazy dla kanałów RGB obrazu Girl 24-bitowego.}%
\end{figure} 

\subsection{Filtracja}
Sekcja ta prezentuje wyniki wyniki zastosowania filtracji dla wybranych obrazów 8-bitowych. Zaprezentowane są próbki oraz widma mocy i fazy przed i po filtracji. 

\subsubsection{Filtr dolnoprzepustowy (górnozaporowy)}
Wyniki zastosowania filtru dolnoprzepustowego na obrazie 8-bitowym Lena zaszumionym szumem jednostajnym 3 są przedstawione poniżej.
\begin{figure}[H]%
    \centering
    \subfloat[Obraz bez filtra]{{\includegraphics[width=3cm,height=3cm,keepaspectratio]{img/transformed/filters/low-pass/low_pass_lena_uniform_3_before.png}}}%
    \qquad
    \qquad
    \subfloat[Widmo mocy bez filtra]{{\includegraphics[width=3cm,height=3cm,keepaspectratio]{img/transformed/filters/low-pass/low_pass_abs_before_lena_8.png}}}%    
    \qquad
    \qquad
    \subfloat[Widmo fazy bez filtra]{{\includegraphics[width=3cm,height=3cm,keepaspectratio]{img/transformed/filters/low-pass/low_pass_phase_before_lena_8.png}}}%    
    \qquad   
    \qquad 
    \subfloat[Obraz z filtrem]{{\includegraphics[width=3cm,height=3cm,keepaspectratio]{img/transformed/filters/low-pass/low_pass_lena_uniform_3_after.png}}}%
    \qquad
    \qquad
    \subfloat[Widmo mocy z filtrem]{{\includegraphics[width=3cm,height=3cm,keepaspectratio]{img/transformed/filters/low-pass/low_pass_abs_after_lena_8.png}}}%
    \qquad
    \qquad
    \subfloat[Widmo fazy z filtrem]{{\includegraphics[width=3cm,height=3cm,keepaspectratio]{img/transformed/filters/low-pass/low_pass_phase_after_lena_8.png}}}%
    \qquad
    \caption{Zastosowanie filtru dolnoprzepustowego na obrazie Lena 8-bitowym zaszumionym szumem jenostajnym; max=10.}%
\end{figure}  

\subsubsection{Filtr górnoprzepustowy (dolnozaporowy)}
Wyniki zastosowania filtru górnoprzepustowego na obrazie 8-bitowym Pentagon są przedstawione poniżej. 
\begin{figure}[H]%
    \centering
    \subfloat[Obraz bez filtra]{{\includegraphics[width=3cm,height=3cm,keepaspectratio]{img/transformed/filters/high-pass/high_pass_pentagon_before.png}}}%
    \qquad
    \qquad
    \subfloat[Widmo mocy bez filtra]{{\includegraphics[width=3cm,height=3cm,keepaspectratio]{img/transformed/filters/high-pass/high_pass_abs_before_pentagon_8.png}}}%    
    \qquad
    \qquad
    \subfloat[Widmo fazy bez filtra]{{\includegraphics[width=3cm,height=3cm,keepaspectratio]{img/transformed/filters/high-pass/high_pass_phase_before_pentagon_8.png}}}%    
    \qquad   
    \qquad 
    \subfloat[Obraz z filtrem]{{\includegraphics[width=3cm,height=3cm,keepaspectratio]{img/transformed/filters/high-pass/high_pass_pentagon_after.png}}}%
    \qquad
    \qquad
    \subfloat[Widmo mocy z filtrem]{{\includegraphics[width=3cm,height=3cm,keepaspectratio]{img/transformed/filters/high-pass/high_pass_abs_after_pentagon_8.png}}}%
    \qquad
    \qquad
    \subfloat[Widmo fazy z filtrem]{{\includegraphics[width=3cm,height=3cm,keepaspectratio]{img/transformed/filters/high-pass/high_pass_phase_after_pentagon_8.png}}}%
    \qquad
    \caption{Zastosowanie filtru górnoprzepustowego na obrazie Pentagon 8-bitowym; min=20.}%
\end{figure}  

\subsubsection{Filtr pasmowoprzepustowy}
Wyniki zastosowania filtru pasmowoprzepustowego na obrazie 8-bitowym Lena są przedstawione poniżej. 
\begin{figure}[H]%
    \centering
    \subfloat[Obraz bez filtra]{{\includegraphics[width=3cm,height=3cm,keepaspectratio]{img/transformed/filters/band-pass/band_pass_lena_before.png}}}%
    \qquad
    \qquad
    \subfloat[Widmo mocy bez filtra]{{\includegraphics[width=3cm,height=3cm,keepaspectratio]{img/transformed/filters/band-pass/band_pass_abs_before_lena_8.png}}}%    
    \qquad
    \qquad
    \subfloat[Widmo fazy bez filtra]{{\includegraphics[width=3cm,height=3cm,keepaspectratio]{img/transformed/filters/band-pass/band_pass_phase_before_lena_8.png}}}%    
    \qquad   
    \qquad 
    \subfloat[Obraz z filtem]{{\includegraphics[width=3cm,height=3cm,keepaspectratio]{img/transformed/filters/band-pass/band_pass_lena_after.png}}}%
    \qquad
    \qquad
    \subfloat[Widmo mocy z filtrem]{{\includegraphics[width=3cm,height=3cm,keepaspectratio]{img/transformed/filters/band-pass/band_pass_abs_after_lena_8.png}}}%
    \qquad
    \qquad
    \subfloat[Widmo faze z filtrem]{{\includegraphics[width=3cm,height=3cm,keepaspectratio]{img/transformed/filters/band-pass/band_pass_phase_after_lena_8.png}}}%
    \qquad
    \caption{Zastosowanie filtru pasmowoprzepustowego na obrazie Lena 8-bitowym; min=10, max=50.}%
\end{figure}  

\subsubsection{Filtr pasmowozaporowy}
Wyniki zastosowania filtru pasmowozaporowego na obrazie 8-bitowym Messer są przedstawione poniżej. 
\begin{figure}[H]%
    \centering
    \subfloat[Obraz bez filtra]{{\includegraphics[width=3cm,height=3cm,keepaspectratio]{img/transformed/filters/band-cut/band_cut_messer_before.png}}}%
    \qquad
    \qquad
    \subfloat[Widmo mocy bez filtra]{{\includegraphics[width=3cm,height=3cm,keepaspectratio]{img/transformed/filters/band-cut/band_cut_abs_before_messer_8.png}}}%    
    \qquad
    \qquad
    \subfloat[Widmo fazy bez filtra]{{\includegraphics[width=3cm,height=3cm,keepaspectratio]{img/transformed/filters/band-cut/band_cut_phase_before_messer_8.png}}}%    
    \qquad   
    \qquad 
    \subfloat[Obraz z filtem]{{\includegraphics[width=3cm,height=3cm,keepaspectratio]{img/transformed/filters/band-cut/band_cut_messer_after.png}}}%
    \qquad
    \qquad
    \subfloat[Widmo mocy z filtrem]{{\includegraphics[width=3cm,height=3cm,keepaspectratio]{img/transformed/filters/band-cut/band_cut_abs_after_messer_8.png}}}%
    \qquad
    \qquad
    \subfloat[Widmo faze z filtrem]{{\includegraphics[width=3cm,height=3cm,keepaspectratio]{img/transformed/filters/band-cut/band_cut_phase_after_messer_8.png}}}%
    \qquad
    \caption{Zastosowanie filtru pasmowozaporowego na obrazie Messer 8-bitowym; min=15, max=30.}%
\end{figure}  

\subsubsection{Filtr z detekcją krawędzi}
//TODO

\subsubsection{Filtr modyfikujący fazę widma transformaty Fouriera}
Wyniki zastosowania filtru modyfikującego fazę widma transformaty Foueriera na obrazie 8-bitowym Bird są przedstawione poniżej. 
\begin{figure}[H]%
    \centering
    \subfloat[Obraz bez filtra]{{\includegraphics[width=3cm,height=3cm,keepaspectratio]{img/transformed/filters/spectrum-modification/spectrum_modification_bird_before.png}}}%
    \qquad
    \qquad
    \subfloat[Widmo mocy bez filtra]{{\includegraphics[width=3cm,height=3cm,keepaspectratio]{img/transformed/filters/spectrum-modification/spectrum_modification_abs_before_bird_8.png}}}%    
    \qquad
    \qquad
    \subfloat[Widmo fazy bez filtra]{{\includegraphics[width=3cm,height=3cm,keepaspectratio]{img/transformed/filters/spectrum-modification/spectrum_modification_phase_before_bird_8.png}}}%    
    \qquad   
    \qquad 
    \subfloat[Obraz z filtem]{{\includegraphics[width=3cm,height=3cm,keepaspectratio]{img/transformed/filters/spectrum-modification/spectrum_modification_bird_after.png}}}%
    \qquad
    \qquad
    \subfloat[Widmo mocy z filtrem]{{\includegraphics[width=3cm,height=3cm,keepaspectratio]{img/transformed/filters/spectrum-modification/spectrum_modification_abs_after_bird_8.png}}}%
    \qquad
    \qquad
    \subfloat[Widmo faze z filtrem]{{\includegraphics[width=3cm,height=3cm,keepaspectratio]{img/transformed/filters/spectrum-modification/spectrum_modification_phase_after_bird_8.png}}}%
    \qquad
    \caption{Zastosowanie filtru modyfikującego fazę widma transformaty Fouriera na obrazie Bird 8-bitowym; k=40, l=200.}%
\end{figure} 

\subsection{Segmentacja}
Sekcja ta prezentuje wyniki wyniki zastosowania segmentacji dla obrazu 24-bitowego Girl. Zaprezentowane są nałożone maski dla różnych parametrów progu i minimalnej ilości pikseli na obszar.

\subsubsection{Metoda rozrostu obszarów}
Wyniki zastosowania metody rozrostu obszarów w celu segmentacji regionów na obrazie 24-bitowym Girl przedstawione są poniżej.
\begin{figure}[H]%
    \centering
    \subfloat[próg=1, minP=200]{{\includegraphics[width=2.5cm,height=2.5cm,keepaspectratio]{img/transformed/segmentation/growing_1_200_girl_24.png}}}%
    \qquad
    \subfloat[próg=5, minP=200]{{\includegraphics[width=2.5cm,height=2.5cm,keepaspectratio]{img/transformed/segmentation/growing_5_200_girl_24.png}}}%    
    \qquad
    \subfloat[próg=10, minP=200]{{\includegraphics[width=2.5cm,height=2.5cm,keepaspectratio]{img/transformed/segmentation/growing_10_200_girl_24.png}}}%    
    \qquad 
    \subfloat[próg=15, minP=200]{{\includegraphics[width=32.5cm,height=2.5cm,keepaspectratio]{img/transformed/segmentation/growing_15_200_girl_24.png}}}%
    \qquad
    \subfloat[próg=1, minP=1000]{{\includegraphics[width=2.5cm,height=2.5cm,keepaspectratio]{img/transformed/segmentation/growing_1_1000_girl_24.png}}}%
    \qquad
    \subfloat[próg=5, minP=1000]{{\includegraphics[width=2.5cm,height=2.5cm,keepaspectratio]{img/transformed/segmentation/growing_5_1000_girl_24.png}}}%    
    \qquad
    \subfloat[próg=10, minP=1000]{{\includegraphics[width=2.5cm,height=2.5cm,keepaspectratio]{img/transformed/segmentation/growing_10_1000_girl_24.png}}}%    
    \qquad 
    \subfloat[próg=15, minP=1000]{{\includegraphics[width=32.5cm,height=2.5cm,keepaspectratio]{img/transformed/segmentation/growing_15_1000_girl_24.png}}}%
    \qquad
    \caption{Zastosowanie metody rozrostu obszarów w celu segmentacji regionów na obrazie 24-bitowym Girl.}%
\end{figure} 

\subsubsection{Metoda podziału obszarów}
Wyniki zastosowania metody podziału obszarów w celu segmentacji regionów na obrazie 24-bitowym Girl przedstawione są poniżej.
\begin{figure}[H]%
    \centering
    \subfloat[próg=1, minP=200]{{\includegraphics[width=2.5cm,height=2.5cm,keepaspectratio]{img/transformed/segmentation/split_merge_1_200_girl_24.png}}}%
    \qquad
    \subfloat[próg=5, minP=200]{{\includegraphics[width=2.5cm,height=2.5cm,keepaspectratio]{img/transformed/segmentation/split_merge_5_200_girl_24.png}}}%    
    \qquad
    \subfloat[próg=10, minP=200]{{\includegraphics[width=2.5cm,height=2.5cm,keepaspectratio]{img/transformed/segmentation/split_merge_10_200_girl_24.png}}}%    
    \qquad 
    \subfloat[próg=15, minP=200]{{\includegraphics[width=32.5cm,height=2.5cm,keepaspectratio]{img/transformed/segmentation/split_merge_15_200_girl_24.png}}}%
    \qquad
    \subfloat[próg=1, minP=1000]{{\includegraphics[width=2.5cm,height=2.5cm,keepaspectratio]{img/transformed/segmentation/split_merge_1_1000_girl_24.png}}}%
    \qquad
    \subfloat[próg=5, minP=1000]{{\includegraphics[width=2.5cm,height=2.5cm,keepaspectratio]{img/transformed/segmentation/split_merge_5_1000_girl_24.png}}}%    
    \qquad
    \subfloat[próg=10, minP=1000]{{\includegraphics[width=2.5cm,height=2.5cm,keepaspectratio]{img/transformed/segmentation/split_merge_10_1000_girl_24.png}}}%    
    \qquad 
    \subfloat[próg=15, minP=1000]{{\includegraphics[width=32.5cm,height=2.5cm,keepaspectratio]{img/transformed/segmentation/split_merge_15_1000_girl_24.png}}}%
    \qquad
    \caption{Zastosowanie metody podziału obszarów w celu segmentacji regionów na obrazie 24-bitowym Girl.}%
\end{figure} 

\section{Dyskusja}
//TODO

\section{Wnioski}
//TODO

\begin{thebibliography}{1}
\bibitem{instruction_pol}\text{$http://ftims.edu.p.lodz.pl/pluginfile.php/19300/mod_resource/content/3/$}\\
\text{$Zadanie2.pdf, 2015$}
\bibitem{instruction_pol}\text{$http://ftims.edu.p.lodz.pl/pluginfile.php/19301/mod_resource/content/0/$}\\
\text{$dft.pdf, 2015$}
\bibitem{instruction_pol}\text{$https://github.com/alisowsk/image-and-sound-processing/blob/master/sprawozdanie/$}\\
\text{$sprawozdanie.pdf, 2015$}
\bibitem{instruction_ang}\text{$http://ics.p.lodz.pl/~tomczyk/available/po\_en/second.html, 2015$}
\bibitem{instruction_latex}\text{$https://en.wikibooks.org/wiki/LaTeX/Mathematics, 2015$}
\bibitem{doc_linear_fitering}\text{$http://lodev.org/cgtutor/fourier.html, 2015$}
\bibitem{doc_linear_fitering}\text{$http://www.doc.ic.ac.uk/~dfg/vision/v02.html, 2015$}
\bibitem{doc_linear_fitering}\text{$http://fourier.eng.hmc.edu/e101/lectures/Image_Processing/node6.html, 2015$}
\bibitem{doc_linear_fitering}\text{$http://users.ecs.soton.ac.uk/msn/book/new\_demo/fourier/, 2015$}
\bibitem{doc_linear_fitering}\text{$https://pl.wikipedia.org/wiki/Algorytm\_Cooleya-Tukeya, 2015$}
\bibitem{doc_linear_fitering}\text{$https://www.cs.cf.ac.uk/Dave/Vision\_lecture/node35.html, 2015$}
\end{thebibliography}

\end{document}
